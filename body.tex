\logosection{\faGraduationCap}{教育背景}
% \setcounter{}{}
\datedline{\textbf{瑞典皇家理工学院 (KTH Royal Institute of Technology)}}{\dateRange{2021.09}{至今}}
\datedline{\tripleInfo{MSc Engineering Design}{\textit{硕士在读}}{机电一体化和嵌入式控制系统系}}{瑞典斯德哥尔摩}
\begin{itemize}
  \item 专业方向:机械电子工程
\end{itemize}

\datedline{\textbf{香港城市大学 (City University of Hong Kong)}}{\dateRange{2017.09}{2021.07}}
\datedline{\tripleInfo{BEng Mechanical Engineering}{\textit{工学学士}}{机械工程系}}{中国香港特别行政区}
\begin{itemize}
  \item 修读课程:控制理论基础、嵌入式系统、机器视觉、机器人学、信号及系统、C\texttt{++}、热力学、流体力学、材料力学、结构力学、CAD/CAM、现代控制理论、微机电系统(MEMS)、生产系统的运行管理等
\end{itemize}

\datedline{\textbf{新加坡国立大学 (National University of Singapore)}}{\dateRange{2020.01}{2020.06}}
\datedline{\tripleInfo{Mechanical Engineering}{\textit{交换生}}{设计与工程学院}}{新加坡}
\begin{itemize}
  \item 修读课程:微机电系统(MEMS)、生物材料、空气动力学、JAVA 
\end{itemize}

% \logosection{\faSuitcase}{工作经历}

% \datedline{\textbf{血汗工厂}}{\dateRange{20xx.xx}{20xx.xx}}
% \datedline{建筑工程一线劳动者}{工地}

% 努力在率先实现社会主义现代化上走在前列

% \begin{itemize}
%   \item 
%   \begin{itemize}
%       \item 
%       \item 
%       \item 
%   \end{itemize}

% \end{itemize}

% \datedline{\textbf{远洋邮轮1号}}{\dateRange{20xx.xx}{20xx.xx}}

% \datedline{\tripleInfo{实习}{水手}{花式摸鱼工程师}}{精神家园}

% 负责策划麦哲伦五百年纪念活动, 坚持不忘初心
% \begin{itemize}
%   \item 精神荷兰人, 海上马车夫复兴者
%   \item 工人阶级的伟大代表 -- 不想当资本家的奴隶不是好工人
% \end{itemize}

\logosection{\faWrench}{项目经历}

\datedline{\textbf{用于挤奶系统的电子真空可调截止阀}}{\dateRange{2022.03}{至今}}
\datedline{\biInfo{\textit{项目实习}}{合作导师:\href{https://www.kth.se/profile/nilsjor}{Nils Jörgensen}}}{瑞典皇家理工学院 {\&} 瑞典利拉伐集团股份有限公司}

\begin{itemize}
\small 
  \item 在考虑食品接触规定的情况下,设计一种由挤奶系统中的传感器进行电子调节奶流量的截止阀
  \item 作为负责人之一对相关领域进行研究,对自行设计的原型机进行实验,分析数据,并总结实验结果
  \item 所设计的系统提高了挤奶系统的稳定性,而且有效消除了系统中的真空度的下降
\end{itemize}

\datedline{\textbf{单目自主飞行无人机}}{\dateRange{2022.01}{2022.06}}
\datedline{\biInfo{\textit{项目课程}}{指导教师:\href{https://www.kth.se/profile/patric/}{Patric Jensfelt}}}{瑞典皇家理工学院}
\begin{itemize}
\small 
  \item 在有限资源下,利用单目自主飞行无人机对已知地图进行搜索,识别未标记的“入侵物”
  \item 研究基于Crazyflie 2.0和VM275T FPV摄像头的无人机定位和设别
  \item 负责整体算法架构和无人机自主飞行的规划、定位和控制
\end{itemize}

% \datedline{\textbf{无人车自主识别}}{\dateRange{2022.01}{2022.06}}
% \datedline{\biInfo{\textit{项目课程}}{指导教师:\href{https://www.kth.se/profile/mjg?l=en}{Martin Edin Grimheden}}}{瑞典皇家理工学院}
% \begin{itemize}
%   \item 作为负责人研究基于Turtlebot3的无人车的规划和对物体的设别
%   \item 负责整体算法架构和无人车的识别、定位、规划和控制
% \end{itemize}
 
\datedline{\textbf{液基气泡发电装置}}{\dateRange{2020.06}{2021.06}}
\datedline{\biInfo{\textit{毕业设计}}{指导教师:\href{https://www.cityu.edu.hk/mne/people/academic-staff/prof-wang-zuankai}{王钻开 WANG Zuankai}}}{香港城市大学}
\begin{itemize}
\small 
  \item 负责研究验证关于利用水中气泡进行发电的基础研究,利用自行设计的装置对液基摩擦纳米发电机(TENG)进行实验,分析结果,并分析实验数据
  \item 所实验的发电装置能利用水中气泡进行发电,具有较强的环境耐受度
\end{itemize}

\datedline{\textbf{风力涡轮机叶片设计和原型制作}}{\dateRange{2021.01}{2021.05}}
\datedline{\biInfo{\textit{课程设计}}{指导教师:\href{https://www.cityu.edu.hk/mne/people/academic-staff/prof-li-ky-lawrence}{LI K.Y.  Lawrence}}}{香港城市大学}
\begin{itemize}
\small 
  \item 利用SolidWorks设计小型风力涡轮机叶片,基于3D打印制造叶片并参与课程比赛,获得第二名
\end{itemize}

\datedline{\textbf{自动导引车AGV导航}}{\dateRange{2021.01}{2021.05}}
\datedline{\biInfo{\textit{课程设计}}{指导教师:\href{https://www.cityu.edu.hk/mne/people/academic-staff/dr-liu-jun}{刘军 LIU Jun}}}{香港城市大学}
\begin{itemize}
\small 
  \item 利用C语言开发自动跟线行驶的引导车,在极限情况下,仅依靠红外线传感器完成路径搜索
  \item 开发基于LPCX13xx和红外线传感器的自动引导车的设计、控制和导航
  \item 负责整体算法架构和引导车的控制和导航
\end{itemize}

\datedline{\textbf{柔性电池的合成和制造}}{\dateRange{2019.04}{2020.01}}
\datedline{\biInfo{\textit{项目实习}}{指导教师:\href{https://www.cityu.edu.hk/mne/people/academic-staff/prof-zhang-kaili}{张开黎 ZHANG Kaili}}}{香港城市大学}
\begin{itemize}
\small 
  \item 对金属氢氧化物进行研究,利用种子辅助合成法制备的柔性电池进行实验,分析数据,并总结实验结果
  \item 所参与实验的新型合成方法和所设计的组装模组皆已申请专利
\end{itemize}

\datedline{\textbf{骨骼愈合的监测和预测}}{\dateRange{2018.09}{2019.01}}
\datedline{\biInfo{\textit{项目实习}}{指导教师:\href{https://www.cityu.edu.hk/bme/yajishen/}{申亚京 SHEN, Yajing}}}{香港城市大学}
\begin{itemize}
\small
  \item 研究多传感器生物体内的协同探测。负责文献综述、样品制备、对制备好的样品进行实验、收集和分析实验结果等。
\end{itemize}

% \newpage
\logosection{\faCogs}{专业技能}
\begin{itemize}[parsep=0.5ex]
  \item 编程语言: C/C\texttt{++} = Matlab > Python > \textsc{java} $\gg$ \texttt{C\#}
  \item 软件: 熟练使用SolidWorks, AutoCAD, MATLAB, Endnote, Zotero, LS-DYNA, ANSYS, Ubuntu, EAGLE, ROS, \LaTeX,Simplify3D, Adobe Premiere Pro, OriginLab, Adobe Illustrator, SPSS, MS Office等,并能快速学习任何已知软件
  \item 硬件:STM32F1/F3,NXP LPC13xx,Arduino/树莓派,3D打印(FDM、SLA),激光切割,车床和铣床,PCB裸板制作(LPKF), PCB焊接与测试,电化学测试平台 (辰华、新威尔、蓝电), 透射电子显微镜TEM(JEOL JEM-2100)/扫描电子显微镜SEM(FEI Quanta 450)
  \item 语言: 中文 - 母语,英语 - 熟练,德语 - 日常读写,瑞典语 - 初学
\end{itemize}

\logosection{\faInstitution}{公开成果及获奖}
\datedline{\textit{发明专利(实质审查),CN 113675454 A,“一种小型柔性电池组装模具”}}{2020.6}
\datedline{\biInfo{\textit{才艺发展奖学金}}{香港特别行政区政府奖学金}}{2020.6}
\datedline{\textit{发明专利(实质审查),CN 14180645 A,“多元金属氢氧化物及其制备方法与应用”}}{2020.2}
\datedline{\biInfo{\textit{教育部第十六届 “挑战杯”大学生课外学术科技作品竞赛}}{国赛二等奖}{队长}}{2019.11}
\datedline{\biInfo{\textit{教育部第五届“互联网 +”大学生创新创业大赛}}{国赛银奖}{核心成员}}{2019.10}
\datedline{\biInfo{\textit{第八届“赢在广州”暨粤港澳大湾区大学生创业大赛}}{银奖}{核心成员}}{2019.7}
\datedline{\biInfo{\textit{第五届香港大学生创新创业大赛}}{科技创新组第二名}{核心成员}}{2019.4}

\logosection{\faBell}{社会服务}
% \datedline{\biInfo{\textbf{共融号-香港文化考察之旅}}{UNLEASH FOUNDATION }{志愿者}}{2018.1}
% \textit{获得聘志发展基金资助,以志愿服务的形式,促进内地学生来港的交流与融入。团队8人服务内地学生200人次,个人服务时长超过100小时。}
\datedline{\biInfo{\textbf{E-buddy}}{TECC (Technology \& Education: Connecting Cultures) 香港分处}{项目负责人}}{\dateRange{2017.09}{2018.06}}
\textit{以志愿服务的形式,通过线上连线的方式,于每周固定时间对留守儿童、孤儿、罕见病儿童、缺乏教育资源地区学生的课外拓展和课程答疑,并组织暑期回访,服务对象达195人次,团队服务时长超过500小时}
\datedline{\biInfo{\textbf{运营部}}{TECC (Technology \& Education: Connecting Cultures) 香港分处}{核心成员}}{\dateRange{2017.09}{2019.01}}
\textit{协调组织运作,进行活动组织策划,组织开展了多项活动,包括NLTP NPO领导力培训、TSI暑期教师学院、YAMP中华民族传承人培养计划等,组织参与人次达到500,参与时长超过200小时}

\logosection{\faBell}{兴趣爱好}
\datedline{\biInfo{Edmond Ko教授杯宿间足球赛}{汇丰业昕堂(Hall2)}{冠军}}{\dateRange{2019.03}{2019.04}}
\datedline{\biInfo{随州市中小学足球比赛}{高中组}{冠军}}{\dateRange{2016.05}{2016.06}}
\datedline{\biInfo{全国中学生生物学联赛}{湖北省}{省二等奖}}{\dateRange{2016.03}{2016.04}}
\par 擅长烹饪,具有6年烹饪经验,掌握中西菜系的各式家常菜、甜点、面食
\par 喜好阅读,已读超过300本各类小说

%%%% 如果多页简历,可以手动在适当位置插入 \newpage 或者 \clearpage 开始新一页